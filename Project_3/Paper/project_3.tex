\documentclass[%
reprint,
amsmath,amssymb,
aps,
]{revtex4-1}
\usepackage{graphicx}% Include figure files
\usepackage{dcolumn}% Align table columns on decimal point
\usepackage{bm}% bold math
\usepackage[utf8]{inputenc}
\usepackage{listings}
\usepackage{amsmath}

\makeatletter
\newcommand*{\rom}[1]{\expandafter\@slowromancap\romannumeral #1@}
\makeatother

\begin{document}
\title{Numerical integration}
\author{Oline A. Ranum}
\affiliation{%
 University of Oslo \\ Institute for physics\\
 olinear@student.matnat.uio.no
}
\date{\today}


\begin{abstract}
	The following experiment was undertaken as a mean to 
\end{abstract}
\maketitle

\section{Introduction}


\section{Theory}
\subsection{Gauss-Legendre quadrature}
\subsection{Gauss-Laguerre quadrature}
\subsection{Monte-Carlo Integration}
\subsection{Parallelization}

\subsection{Wave function} \noindent 
The single-particcle wave function for an electron $i$ in the $1s$ state is given in terms of a dimensionless variable 
\begin{equation*}
	 {\bf r}_i =  x_i {\bf e}_x + y_i {\bf e}_y +z_i {\bf e}_z 
\end{equation*}
as 
\begin{equation*}
	\psi_{1s}({\bf r}_i)  =   e^{-\alpha r_i},
\end{equation*}
where $\alpha$ is a parameter and 
\begin{equation*}
	r_i = \sqrt{x_i^2+y_i^2+z_i^2}
\end{equation*}
For a helium atom, $\alpha = 2$ with $Z = 2$. The ansatz for the wave function for two electrons is then given by the product of two so-called 1s wave functions
\begin{equation*}
	\Psi({\bf r}_1,{\bf r}_2)  =   e^{-\alpha (r_1+r_2)}
\end{equation*}
This ansatz does not yield a closed-form or analytical solution to Schrodinger's equation for two interacting electrons in the helium atom. \\
We are then left with the following integral yielding the quantum mechanical expectation value of the correlation energy between two electrons which repel each other via the classical Coulomb interaction
\begin{equation}
	   \langle \frac{1}{|{\bf r}_1-{\bf r}_2|} \rangle =
	\int_{-\infty}^{\infty} d{\bf r}_1d{\bf r}_2  e^{-2\alpha (r_1+r_2)}\frac{1}{|{\bf r}_1-{\bf r}_2|}
\end{equation}
This wavefunction is not normalized, so the integral has the closed form solution 
\begin{equation}
\langle \frac{1}{|{\bf r}_1-{\bf r}_2|} \rangle = \dfrac{5}{16^2}\pi^2
\end{equation}
\vspace{10mm}
\section{Method}
\subsection{1. Integration}
Integrate in a brute force manner a six dimensional integral which is used to determine the ground state correlation energy between two electrons in a helium atom. 

We assume that the wave function of each electron can be modelled like the single-particle wave function of an electron in the hydrogen atom. 

\section{Result}
\section{Discussion}
\section{Conclusion }

\end{document}