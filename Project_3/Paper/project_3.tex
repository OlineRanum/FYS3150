\documentclass[%
reprint,
amsmath,amssymb,
aps,
]{revtex4-1}
\usepackage{graphicx}% Include figure files
\usepackage{dcolumn}% Align table columns on decimal point
\usepackage{bm}% bold math
\usepackage[utf8]{inputenc}
\usepackage{listings}
\usepackage{amsmath}

\makeatletter
\newcommand*{\rom}[1]{\expandafter\@slowromancap\romannumeral #1@}
\makeatother

\begin{document}
\title{Numerical integration}
\author{Oline A. Ranum}
\affiliation{%
 University of Oslo \\ Institute for physics\\
 olinear@student.matnat.uio.no
}
\date{\today}


\begin{abstract}
	The following experiment was undertaken as a mean to 
\end{abstract}
\maketitle

\section{Introduction}


\section{Theory}



\subsection{Numerical integration \& the Gaussian quadrature} \noindent 
The basic idea behind all numerical integration methods is to approximate the integral 
\begin{equation}\label{numint}
	I = \int_{a}^{b}f(x)dx \approx \sum_{i=1}^{N} \omega_if(x_i)
\end{equation}
where $\omega$ and $x$ are the weights and the chosen mesh points, respectively.

Interpolatory quadrature rules, such as Simpson's- or the trapezoidal rule, are based on the assumption that the quadrature points, or nodes, are preassigned equidistantly or with a fixed distribution. The Gaussian quadrature (hereafter GQ) is based on the notion first made by Gauss, that a suitable variation of the nodes would in general lead to a better accuracy. As such, many variations and generalizations of the Gaussian formulas have been developed on the form equation \ref{numint}, where the weights $\omega_i$ are positive zeros of certain orthogonal polynomials and the nodes $x_i$ are distinct points in the interval $x\in[-1,1]$ [Kythe \& Schaferkotter, 2005]. 

\subsection{Orthogonal polynomials} \noindent 
A set of polynomials $\{p_i\}$ with degree $i$ is called orthogonal with respect to the inner-product if $<p_i, p_j> = 0$ for $i\not = j$ on a finite or infinite interval $[a,b]$. That is, if the powers are orthonormalized, one should obtain a unique set of polynomials $p_i(x)$ of degree $n$ such that
\begin{equation}
	\int_{a}^{b} \omega(x)p_i(x)p_j(x)dx = \delta_{ij} = \left\{
	\begin{array}{ll}
	1 & i = j\\
	0 & i \not =j 
	\end{array} \right\}
\end{equation}
where $\delta_{ij}$ is the Kronecker delta. Orthogonal polynomials defined in this manner also satisfy a discrete orthogonality. 
\begin{equation}
	\sum_{i=1}^{n+1} \omega_i p_j(x_i)p_k(x_i) = \delta_{jk}
\end{equation}
See Kythe \& Schaferkotter 2005 for further elaboration of this property. 

\subsection{Gauss-Legendre quadrature}
\subsection{Gauss-Laguerre quadrature}
\subsection{Monte-Carlo Integration}
\subsection{Parallelization}

\subsection{Wave function} \noindent 
The single-particcle wave function for an electron $i$ in the $1s$ state is given in terms of a dimensionless variable 
\begin{equation*}
	 {\bf r}_i =  x_i {\bf e}_x + y_i {\bf e}_y +z_i {\bf e}_z 
\end{equation*}
as 
\begin{equation*}
	\psi_{1s}({\bf r}_i)  =   e^{-\alpha r_i},
\end{equation*}
where $\alpha$ is a parameter and 
\begin{equation*}
	r_i = \sqrt{x_i^2+y_i^2+z_i^2}
\end{equation*}
For a helium atom, $\alpha = 2$ with $Z = 2$. The ansatz for the wave function for two electrons is then given by the product of two so-called 1s wave functions
\begin{equation*}
	\Psi({\bf r}_1,{\bf r}_2)  =   e^{-\alpha (r_1+r_2)}
\end{equation*}
This ansatz does not yield a closed-form or analytical solution to Schrodinger's equation for two interacting electrons in the helium atom. \\
We are then left with the following integral yielding the quantum mechanical expectation value of the correlation energy between two electrons which repel each other via the classical Coulomb interaction
\begin{equation}
	   \langle \frac{1}{|{\bf r}_1-{\bf r}_2|} \rangle =
	\int_{-\infty}^{\infty} d{\bf r}_1d{\bf r}_2  e^{-2\alpha (r_1+r_2)}\frac{1}{|{\bf r}_1-{\bf r}_2|}
\end{equation}
This wavefunction is not normalized, so the integral has the closed form solution 
\begin{equation}
\langle \frac{1}{|{\bf r}_1-{\bf r}_2|} \rangle = \dfrac{5}{16^2}\pi^2
\end{equation}
\subsubsection*{Numerical integration of the wave function}
For all practical purposes the lower- and upper infinite integration limits can be substituted by a finitie nummber $\lambda$. The single-particle wave function $e^{-\alpha r_i}$ is more or less zero at a finite value $r_i\approx \lambda$, therefore the limitis $-\infty$ and $\infty$ can be substituted by $-\lambda$ and $\lambda$ respectively. 
\vspace{10mm}
\section{Method}
\subsection{1. Integration}
Integrate in a brute force manner a six dimensional integral which is used to determine the ground state correlation energy between two electrons in a helium atom. 

We assume that the wave function of each electron can be modelled like the single-particle wave function of an electron in the hydrogen atom. 

\subsection*{Gauss-Legendre Quadrature}
I use Gauss-Legendre quadrature to compute a six dimensional integral over all Cartesian spatial variables $x_1, y_1, z_1, x_2,y_2,z_2$. To find the integration limits for this numerical integration, I plot the single-particle wave function and locate the value where the wave function is smaller than $10^{-5}$. I then preforme an integral over various N to find how many mesh points is needed before the results converges at the level of the thir leading digit. 

\section{Result}
\section{Discussion}
\section{Conclusion }

\section{Referances}
[1] Handbook of computational methods for integration, P. K. Kythe and M. R. Schaferkotter, Chapman \& Hall/CRC, Boca Raton Florida, 2005.

\end{document}