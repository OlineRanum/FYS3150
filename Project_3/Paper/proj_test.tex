\documentclass[a4paper,twocolumn]{article}
\usepackage[brazilian]{babel}
\usepackage[utf8]{inputenc}
\usepackage{graphicx}
\usepackage{indentfirst}
\usepackage{float}
\usepackage{fancyhdr}
\usepackage{amssymb}
\usepackage{amsmath}
\usepackage[top=2.5cm,left=1.5cm,right=1.5cm,bottom=3cm]{geometry}
\newcommand{\HRule}{\rule{\linewidth}{0.5mm}}



\begin{document}
\twocolumn[\begin{@twocolumnfalse}

\begin{center}
\large\textbf{Kit de Automação de Janelas}
\end{center}

\begin{center}
\small{$^1$Abner Lucas, $^2$Daniel Braga , $^3$Gabriel Sales, $^4$Jakyson Dias, $^4$Lucas Andrade ,$^5$Vinícius Neves }\\
\end{center}

\begin{center}
\small{$^1$Departamento de Física, Universidade Federal de Lavras, C.P. 3037, 37200–000, Lavras, MG, Brasil.\\
$^2$Turma 31A do curso de Engenharia ABI, Universidade Federal de Lavras, C.P. 3037, 37200–000, Lavras, MG, Brasil.}
\end{center}

\begin{center}
\textit{\today}
\end{center}

\HRule \\
O conceito de automação residencial é definido como o conjunto de serviços proporcionados por sistemas tecnológicos integrados, sendo a melhor maneira de satisfazer as  necessidades  básicas  de  segurança,  comunicação, gestão energética e conforto de uma habitação. Seguindo essa concepção, surgiu-se a ideia da criação de um Kit automatizado para janelas utilizando a plataforma Arduíno, visando a solução de problemas do dia a dia como o transtorno causado pela chuva e principalmente, gerando praticidade e comodidade para os cidadãos.   \\
\HRule
\vspace{1cm}

\end{@twocolumnfalse}]



\end{document}