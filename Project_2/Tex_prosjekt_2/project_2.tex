\documentclass[%
reprint,
amsmath,amssymb,
aps,
]{revtex4-1}
\usepackage{graphicx}% Include figure files
\usepackage{dcolumn}% Align table columns on decimal point
\usepackage{bm}% bold math
\usepackage[utf8]{inputenc}
\usepackage{listings}
\usepackage{amsmath}

\makeatletter
\newcommand*{\rom}[1]{\expandafter\@slowromancap\romannumeral #1@}
\makeatother

\begin{document}
\title{Eigenvalue problems}
\author{Torstein S. Ølberg, Ada M. Veddegjerde and Oline A. Ranum}
\affiliation{%
 \textnormal{Universitetet i Oslo, Institutt for fysikk}\\
 olinear@student.matnat.uio.no : torsteol@student.matnat.uio.no
}
\date{\today}


\begin{abstract}
	The following experiment was undertaken as
\end{abstract}
\maketitle

\section*{Introduction}


\section*{Theory}

\section*{Method}
\subsection{The Harmonic Oscillator} \noindent 
The classical harmonic oscillator is described by the following equation of motion
\begin{equation*}
	\gamma \frac{d^2 u(x)}{dx^2} = -F u(x)
\end{equation*}

where $u(x)$ is the vertical displacement of the system in the $y$ direction.

\subsection{Unitary transformations} \noindent 
A unitary transformation is a transformation that preserves the orthogonality and inner product. Given a basis of vectors $\vec{v_i}$ ,
\begin{equation*}
	\mathbf{v}_i = \begin{bmatrix} v_{i1} \\ \dots \\ \dots \\v_{in} \end{bmatrix}
\end{equation*}
where the basis is orthogonal 
\begin{equation*}
	\mathbf{v}_j^T\mathbf{v}_i = \delta_{ij}
\end{equation*}
Given the unitary transformation 
\begin{equation*}
	\mathbf{w}_i=\mathbf{U}\mathbf{v}_i
\end{equation*}
It can be shown that the transformation preserves orthogonality
\begin{align*}
	\mathbf{w}=\mathbf{U}\mathbf{v} &\implies \mathbf{w}^T=\mathbf{v}^T\mathbf{U}^T \\ &\implies \mathbf{w}^T\mathbf{w} = \mathbf{v}^T\mathbf{U}^T\mathbf{U}\mathbf{v} = \mathbf{v}^T\mathbf{v} = \delta
\end{align*}
To show that it preserves the inner product we define 
\begin{align*}
	\mathbf{a} &= U\mathbf{b} \\
	\rightarrow \mathbf{a} &= \mathbf{b}^TU^T \\
	&\textnormal{We then take the inner product:} \\
	 \mathbf{a}^T\mathbf{w} &= \mathbf{b}^TU^TU\mathbf{v} = \mathbf{b}^T\mathbf{v}
\end{align*}
Where we have showed that the unitary transformation does not affect the inner product. 
\section*{Results}

\section*{Discussion} 

\section*{Conclusion}


\newpage .
\newpage 
\onecolumngrid
\section*{Bibliography}
\noindent $[1]$ \\ 
$[2]$
\section*{Appendix A}
For the code used for calculation our results, visit
\url{https://github.com/OlineRanum/FYS3150_Project_1}

\end{document}