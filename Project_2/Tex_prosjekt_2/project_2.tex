\documentclass[%
reprint,
amsmath,amssymb,
aps,
]{revtex4-1}
\usepackage{graphicx}% Include figure files
\usepackage{dcolumn}% Align table columns on decimal point
\usepackage{bm}% bold math
\usepackage[utf8]{inputenc}
\usepackage{listings}
\usepackage{amsmath}

\makeatletter
\newcommand*{\rom}[1]{\expandafter\@slowromancap\romannumeral #1@}
\makeatother

\begin{document}
\title{Eigenvalue problems}
\author{Torstein S. Ølberg, Ada M. Veddegjerde and Oline A. Ranum}
\affiliation{%
 \textnormal{Universitetet i Oslo, Institutt for fysikk}\\
 olinear@student.matnat.uio.no : torsteol@student.matnat.uio.no
}
\date{\today}


\begin{abstract}
	The following experiment was undertaken as
\end{abstract}
\maketitle

\section*{Introduction}


\section*{Theory}

\subsection*{Eigensystem solvers}
For a discretized systems it is possible to directly solve eigenvalue problems using similarity transformations. For instance using the Jacobi rotation method. The jacobi method transforms the matrix into a tridiagonal form using Housholder's algorithm. Given a symmetric matrix $A\in \mathbb{R}^{n\times n}$, we know that there exsists a orthogonal matrix S so that 
\begin{equation*}
	\mathbf{S}^T\mathbf{A}\mathbf{S} = \mathbf{D}
\end{equation*}
where 
\begin{equation*}
	\mathbf{D} = \begin{bmatrix}
	\lambda_1 & 0 & 0 & \dots & 0 \\
	0 & \lambda_2 & 0 & \dots & 0 \\
	\vdots & \vdots & \vdots &\vdots&\vdots\\
	0 & 0 & 0 &\dots & \lambda_n
	\end{bmatrix}
\end{equation*}
and $\lambda_i$ are the eigenvalues of $A$. Thus one can preform a series of similarity transformations on $\mathbf{A}$ in order to reduce it to a diagonal form. Matrix $\mathbf{B}$ is said to be a similarity transform of $\mathbf{A}$ if 
\begin{equation} \label{eq2}
	\mathbf{B} = \mathbf{S}^T\mathbf{A}\mathbf{S}, \hspace{3mm} \textnormal{where} \hspace{3mm} \mathbf{S}^T\mathbf{S} = \mathbf{S}^{-1}\mathbf{S} = \mathbf{I}
\end{equation}
A similarity transform yields a matrix with the same eigenvalues, but in general different eigenvectors. \\
Initially, one has the eigenvalue problem 
\begin{equation}
	\mathbf{A}\mathbf{x} = \lambda\mathbf{x}
\end{equation}
and a similarity matrix as defined in \ref{eq2}. One then subsequently apply a similarity transform so that 
\begin{equation}
	\mathbf{S}_N^T...\mathbf{S}_1^T\mathbf{A}\mathbf{S}_1...\mathbf{S}_N = \mathbf{D}
\end{equation}
Where the diagonal elements of $\mathbf{D}$ are the eigenvalues of $\mathbf{A}$, this method is better known as Jacobi's method [M. Jensen].

\subsubsection*{Jacobi's method} 
Jacobi's method conciders an $n\times n$ orthogonal transformation matrix on the form 
\begin{equation*}
\mathbf{S} = \begin{bmatrix}
1 & 0  & \dots & 0 &\vdots &0 \\
0 & 1  & \dots & 0 & \vdots & 0 \\
\vdots & \vdots &\vdots&\vdots&\vdots&\vdots\\
0 & 0 & \cos{\theta} & 0 &\vdots & \sin{\theta} \\
0  & 0 & 0 & 1 &\vdots & 0 \\
 \vdots & \vdots &\vdots&\vdots&\vdots&\vdots\\
 0 & 0 &\dots & -\sin{\theta} & \vdots & \cos{\theta}
\end{bmatrix}
\end{equation*}
where $\mathbf{S} = \mathbf{S}^-1$. This matrix makes a plane rotation of an angle $\theta$ in the Euclidean n-dimensional space. 

\subsection{The Harmonic Oscillator} \noindent 
The classical harmonic oscillator is described by the following equation of motion
\begin{equation*}
	\gamma \frac{d^2 u(x)}{dx^2} = -F u(x)
\end{equation*}

where $u(x)$ is the vertical displacement of the system in the $y$ direction. 

\subsection{Unitary transformations} \noindent 
A unitary transformation is a transformation that preserves the orthogonality and inner product. Given a basis of vectors $\vec{v_i}$ ,
\begin{equation*}
	\mathbf{v}_i = \begin{bmatrix} v_{i1} \\ \dots \\ \dots \\v_{in} \end{bmatrix}
\end{equation*}
where the basis is orthogonal 
\begin{equation*}
	\mathbf{v}_j^T\mathbf{v}_i = \delta_{ij}
\end{equation*}
Given the unitary transformation 
\begin{equation*}
	\mathbf{w}_i=\mathbf{U}\mathbf{v}_i
\end{equation*}
It can be shown that the transformation preserves orthogonality
\begin{align*}
	\mathbf{w}=\mathbf{U}\mathbf{v} &\implies \mathbf{w}^T=\mathbf{v}^T\mathbf{U}^T \\ &\implies \mathbf{w}^T\mathbf{w} = \mathbf{v}^T\mathbf{U}^T\mathbf{U}\mathbf{v} = \mathbf{v}^T\mathbf{v} = \delta
\end{align*}
To show that it preserves the inner product we define 
\begin{align*}
	\mathbf{a} &= U\mathbf{b} \\
	\rightarrow \mathbf{a} &= \mathbf{b}^TU^T \\
	&\textnormal{We then take the inner product:} \\
	 \mathbf{a}^T\mathbf{w} &= \mathbf{b}^TU^TU\mathbf{v} = \mathbf{b}^T\mathbf{v}
\end{align*}
Where we have showed that the unitary transformation does not affect the inner product. 


\section*{Method}
\subsection{Jacobi's method for Eigensolvers}
\noindent When implementing Jacobi's method we apply the following algorithm to a matrix A:
\begin{align*}
	&b_{ii} = a_{ii} \hspace{2mm}& i\not=k &\hspace{2mm} i\not=l\\
	&b_{ik} = a_{ik}\cos\theta -a_{il}\sin\theta \hspace{2mm}& i\not=k &\hspace{2mm} i\not=l \\
	&b_{il}  = a_{il}\cos\theta -a_{ik}\sin\theta \hspace{2mm}& i\not=k &\hspace{2mm} i\not=l\\
	&b_{kk} = a_{kk}\cos^2\theta -2a_{kl}\cos\theta\sin\theta +a_{ll}\sin^2\theta&&\\
	&b_{ll} = a_{ll}\cos^2\theta -2a_{kl}\cos\theta\sin\theta +a_{kk}\sin^2\theta&& \\
	&b_{kl} = (a_{kk}-a_{ll})\cos\theta\sin\theta+a_{kl}(\cos^2\theta -\sin^2\theta)\\
\end{align*}

\section*{Results}

\section*{Discussion} 

\section*{Conclusion}


\newpage .
\newpage 
\onecolumngrid
\section*{Bibliography}
\noindent $[1]$ \\ 
$[2]$
\section*{Appendix A}
For the code used for calculation our results, visit
\url{https://github.com/OlineRanum/FYS3150_Project_1}

\end{document}