\documentclass[%
reprint,nofootinbib,
amsmath,amssymb,
aps,
]{revtex4-1}
\usepackage{graphicx}% Include figure files
\usepackage{dcolumn}% Align table columns on decimal point
\usepackage{bm}% bold math
\usepackage[utf8]{inputenc}
\usepackage{listings}
\usepackage{amsmath}
\usepackage{physics}
\usepackage{booktabs}
\usepackage{float}
\usepackage[bottom]{footmisc}
\usepackage{scrextend}

\usepackage{caption}
\captionsetup{justification=raggedright,singlelinecheck=false}
\usepackage{subcaption}
\graphicspath{ {./figures/} }




\bgroup
\def\arraystretch{1.3}
\newcommand{\HRule}{\rule{\textwidth}{0.5mm}}
\makeatletter
\newcommand*{\rom}[1]{\expandafter\@slowromancap\romannumeral #1@}
\makeatother

\newcommand{\aegis}[0] {AE$\bar{\textnormal{g}}$IS}
\newcommand{\h}[0] {$\bar{\textnormal{H}}$}



\begin{document}
\onecolumngrid

\begin{center}
	\large\textbf{Building a model for the solar system \\ using ordinary differential equations}
\end{center}
\vspace{5mm}

\begin{center}
	\small{$^1$ Oline A. Ranum}\\
\end{center}

\begin{center}
	\small{$^1$ University of Oslo, Institute of physics, 
		olinear@student.matnat.uio.no}
\end{center}

\begin{center}
	\textit{\today}
\end{center}
\vspace{7mm}
\noindent 
\HRule \vspace{2mm}\\
	 In this project code is developed for simulating the solar system using the Verlet Algorithm 
\noindent 
\vspace{1.5mm}  \\
\HRule
\vspace{.2cm}


\section{Introduction} \noindent 
\vspace{3mm}
\twocolumngrid
\noindent 
\newpage. \newpage .\newpage 

\section{Theory} \noindent 
\subsection{Newton's law of gravitation}
Newton's law of gravitation is given by a force $F_G$
\begin{equation}
	F_G = \frac{GM_{\odot}M_{\mathrm{Earth}}}{r^2} = \frac{M_{\mathrm{Earth}}v^2}{r}
\end{equation}
where $M_{\odot}$ is the mass of the Sun and $M_{\mathrm{Earth}}$ is the mass of the Earth. $G$ is the gravitational constant, $r$ is the distance between the Sun and the Earth and $v$ is the velocity of Earth. This implies that
\begin{equation}\label{motion1}
	v^2r = GM_{\odot} = 4\pi^2\dfrac{(\textnormal{AU})^3}{(\textnormal{yr})^2}
\end{equation}



\subsection{Co-planar motion} \noindent 
If it is assumed that the orbit of the Sun and the Earth is co-planer in the xy-plane, one can employ Newton's law of motion to derive the following equations
\begin{align}
	\frac{d^2x}{dt^2}&=\frac{F_{G,x}}{M_{\mathrm{Earth}}} \\ & \nonumber \\ 
	\frac{d^2y}{dt^2}&=\frac{F_{G,y}}{M_{\mathrm{Earth}}}
\end{align}
where $F_{G,x}$ and $F_{G,y}$ are the $x$ and $y$ components of the gravitational force. 

\subsection{Planetary sizes}

\begin{table}
	\caption{\textit{The mass of the Sun and the masses of all relevant planets and their distances from the sun listed in units of kg and AU, $1$ AU = $1.5\times 10^{11}$ m. }}
	\begin{tabular}{lll}
		\hline
		\multicolumn{1}{l}{ Planet } \hspace{5mm}& \multicolumn{1}{l}{ Mass [kg] }  \hspace{21mm}& \multicolumn{1}{l}{ Distance to  sun [AU] } \\
		\hline
		Earth   & $M_{\mathrm{Earth}}=6\times 10^{24}$      & 1                    \\
		Jupiter & $M_{\mathrm{Jupiter}}=1.9\times 10^{27}$  & 5.20                 \\
		Mars    & $M_{\mathrm{Mars}}=6.6\times 10^{23}$    & 1.52              \\
		Venus   & $M_{\mathrm{Venus}}=4.9\times 10^{24}$    & 0.72                 \\
		Saturn  & $M_{\mathrm{Saturn}}=5.5\times 10^{26}$  & 9.54                 \\
		Mercury & $M_{\mathrm{Mercury}}=3.3\times 10^{23}$ & 0.39                 \\
		Uranus  & $M_{\mathrm{Uranus}}=8.8\times 10^{25}$   & 19.19                \\
		Neptun  & $M_{\mathrm{Neptun}}=1.03\times 10^{26}$  & 30.06            \\ && \\ 
		Sun & $M_{\mathrm{sun}}=M_{\odot}=2\times 10^{30}$& - \\ 
		\hline
	\end{tabular}
\end{table}

\section{Numerical algorithms}
Euler's forward algorithm states that
\begin{equation}
	\dfrac{du(x)}{dt} = \dfrac{u(x+h)-u(x)}{h} + \order{}
\end{equation}

\section{Method} \noindent 
It is assumed that the Sun has a mass which is much larger than the mass of the Earth, so that the motion of the Sun is neglected. In order to discretize equation \ref{motion1} a discrete time axis $t\in[t_0, t_{f}]$ of N points is defined, where $t_i = t_0 + ih$ for $i\in[0,N]$. This implies that $h = (t_f-t_0)/N$. The discretized positions and velocities are then functions of the discretized time $x_i = x(t_i)$ and $v_i = v(t_i)$, and equivalently for $y$. The acceleration is then $a_{xi} = a_x(x_i,y_i)$. \\
The differential equation \ref{motion1} is then solved by Euler's forward algorithm and the velocity Verlet method. 


 
\textbf{Compute the motion of the earth using different methods for solving odrinary differential equations} \\ 
\subsubsection{Initial conditions} \noindent 
The NASA jet propulsion laboratory's HORIZONS Web-Interface is used to generate a cartesian state vector table of any object with respect to any major body, to extract initial conditions for each planet. 
\section{Results} \noindent 
\section{Discussion} \noindent
\section{Conclusion} \noindent  




\onecolumngrid 
\newpage 
\section{References}
\bibliography{mybib}{}
\bibliographystyle{plain}

\section{Appendix}

\end{document}